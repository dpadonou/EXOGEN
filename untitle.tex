\documentclass[12pt,a4paper]{article}
\usepackage[utf8]{inputenc}
\usepackage[french]{babel}
\usepackage[T1]{fontenc}
\usepackage{amsmath}
\usepackage{amsfonts}
\usepackage{amssymb}
\usepackage[bloc,completemulti,nowatermark]{automultiplechoice}    
\usepackage{multicol}
\usepackage{moreverb}

% Appel du package pythontex 
\usepackage{pythontex}
%

\begin{document}
\begin{pycode}

import random
import Util
from Tree import Tree
from Story import Story
import copy
def resolve():
	tree = Tree()
	
	#modifier le seed (graine) pour changer les nombre l'ordre des nombres générés
	vals = Util.generate_number(1, 20)
	_vals = Util.generate_number(1,20)
	
	#suppression des crochets
	__vals = ",".join(map(str, vals))
	
	
	#construction de l'arbre
	for i in vals:
		tree[i] = str(i)
		
	#generation du graphe
	#tree.generateGraph("graph")
	
	
	#recupperer le root
	root = tree.root.keys
	_root = ','.join(map(str,root))
	for i in root :
		_vals.remove(i)
	
	#compter le nombre de noeud distinct pour recuperer une valeur
	nb,node = tree.search(17)
	_nb = Util.generate_number(nb, 5)
	
	#supprimer le nombre noeud si existant dans liste generée
	if nb in _nb :
		_nb.remove(nb)
		
		
	#compter le nombre de noeud distinct pour recuperer une valeur
	nmb,node = tree.search(21)
	_nmb = Util.generate_number(nmb, 5)
	
	#supprimer le nombre noeud si existant dans liste generée
	if nmb in _nmb :
		_nmb.remove(nmb)
		
	#compter le nombre de noeud distinct pour recuperer les valeurs de l'intervalle
	trouve, inter_val, leaves = tree.search_range(8,16)
	_inter_val = Util.generate_number(inter_val, 7)
	
	#supprimer le nombre noeud si existant dans liste generée
	if inter_val in _inter_val:
		_inter_val.remove(inter_val)
	
	#generation de l'histoire
	hist = Story(15, 3, 3, True, 'NR')
	list_hist = []
	_sol = []
	for operation in hist.operations:
		list_hist.append(operation.show())
	lt = " ".join(map(str, list_hist))
	
	sol, at=hist.solve_history()
	for op in sol:
		_sol.append(op.show())
	__sol = " ".join(map(str,_sol))
	
	
	cp=copy.deepcopy(sol)
	i_, j_ = Util.random_choice(cp)
	cp_=Util.swap_elmts(cp, i_, j_)
	_cp_=Util.display(cp_)
	
	_cp=copy.deepcopy(sol)
	i1, i2 = Util.random_choice(_cp)
	_cp__=Util.swap_elmts(_cp, i1, i2)
	__cp=Util.display(_cp__)

	return __vals,nb,_nb,inter_val, _inter_val, nmb, _nmb,_root,_vals,lt, __sol, _cp_,__cp
	

\end{pycode}
\AMCrandomseed{1237893}

\element{amc}{
  \begin{questionmult}{creation-arbre}\bareme{mz=2}
Construire un arbre B+ d'ordre 2 (donc au plus 4 éléments par noeud), au fur et à mesure des insertions suivantes :

\pyc{vals,nb,_nb,inter_val, _inter_val, nmb, _nmb, _root,_vals,lt, __sol, i, j=resolve()}
$$\py{vals}$$

Que contient la racine de cet arbre à la fin de ces insertions?
    \begin{multicols}{2}
      \begin{reponses}
        \bonne{\py{_root}}
        \mauvaise{\py{_vals[1]}}
        \mauvaise{\py{_vals[7]}}
        \mauvaise{\py{_vals[9]},\py{_vals[4]},\py{_vals[5]}}
\end{reponses}
    \end{multicols}
  \end{questionmult}
}

\element{amc}{
\begin{question}{dessin-arbre}
  Dessinez l'arbre B+ \textbf{final} ci-dessous (ne \textbf{pas} cocher les cases f, p, j ci-contre)
  \AMCOpen{lines=7, dots=false}{\wrongchoice[F]{f}\scoring{0}\wrongchoice[P]{p}\scoring{0}\correctchoice[J]{j}\scoring{0}}
\end{question}
}

\element{amc}{
  \begin{questionmult}{acces-noeud}\bareme{mz=3}
On suppose que les n-uplets sont stockés aux feuilles de l'arbre B+. Combien faut-il lire  de pages disque distinctes (donc, ne pas compter deux fois la même page) pour
rechercher l'élément 17  ?
    \begin{multicols}{2}
      \begin{reponses}
        \bonne{\py{nb}}
        \mauvaise{\py{_nb[1]}}
        \mauvaise{\py{_nb[3]}}
        \mauvaise{Aucune}
      \end{reponses}
    \end{multicols}
  \end{questionmult}
}

\element{amc}{
  \begin{questionmult}{acces-noeud-2}\bareme{mz=3}
De même, combien faut-il lire de pages disque distinctes pour tester la présence de l'élément 21 ?
    \begin{multicols}{2}
      \begin{reponses}
        \bonne{\py{nmb}}
        \mauvaise{\py{_nmb[0]}}
        \mauvaise{\py{_nmb[2]}}
        \mauvaise{\py{_nmb[1]}}
      \end{reponses}
    \end{multicols}
  \end{questionmult}
}

\element{amc}{
  \begin{questionmult}{acces-noeud-3}\bareme{mz=3}
De même, combien faut-il lire de pages disque distinctes pour accéder aux n-uplets de l'intervalle $[8,16]$ ?
    \begin{multicols}{2}
      \begin{reponses}
        \bonne{\py{inter_val}}
        \mauvaise{\py{_inter_val[0]}}
        \mauvaise{\py{_inter_val[4]}}
        \mauvaise{\py{_inter_val[1]}}
      \end{reponses}
    \end{multicols}
  \end{questionmult}
}

\element{amc}{
  \begin{questionmult}{plan-1}\bareme{mz=3}
On observe le plan d'exécution suivant :\\

{\tt
SELECT STATEMENT\\
|  1 |     TABLE ACCESS BY INDEX ROWID\\
|  2 |     ..... INDEX UNIQUE SCAN\\
}


Quelle est la requête SQL pouvant donner ce plan ?
%    \begin{multicols}{2}
      \begin{reponses}
        \bonne{select * from Client where Client.id=1254}
        \mauvaise{select * from Client, Commande where Client.id=Commande.id and Client.id=5}
        \mauvaise{select * from Client where Client.id < 5}
        \mauvaise{select index from access where index='unique'}
      \end{reponses}
%    \end{multicols}
  \end{questionmult}
}

\element{amc}{
  \begin{questionmult}{plan-2}\bareme{mz=3}
On observe le plan d'exécution suivant :\\

{\tt
SELECT STATEMENT\\
|  1 |	MERGE JOIN\\
|  2 |  .....SORT JOIN\\
|  3 |  ..........TABLE ACCESS FULL\\
|  4 |  .....SORT JOIN\\
|  5 |  ...............NESTED LOOP\\
|  6 |  ..................... TABLE ACCESS FULL\\
|  7 |  ..................... TABLE ACCESS FULL\\
}


Quelle est la requête SQL pouvant donner ce plan ?
%    \begin{multicols}{2}
      \begin{reponses}
        \mauvaise{select nom from Client order by montant asc}
        \mauvaise{select full from nested1, nested2 where loop is not null}
        \bonne{select * from Client, Commande, Produit where Client.id=Commande.cid and Commande.pid=Produit.id}
        \mauvaise{select * from Client natural join Commande where Commande.id=513}
      \end{reponses}
%    \end{multicols}
  \end{questionmult}
}

\element{amc}{
  \begin{question}{serialisabilite}\bareme{mz=3}

Soit l'histoire suivante reçue par un ordonnanceur :
$$
\py{lt}
$$

 Cette histoire est-elle sérialisable ?
 
      \begin{reponses}
        \mauvaise{oui}
        \bonne{non}
      \end{reponses}

  \end{question}
}

\element{amc}{
  \begin{questionmult}{recouvrabilite}\bareme{mz=3}

 Cette histoire est :
 
      \begin{reponses}
        \mauvaise{recouvrable, évitant les annulations en cascade et stricte}
        \mauvaise{recouvrable, évitant les annulations en cascade mais pas stricte}
        \bonne{pas recouvrable}
        \mauvaise{stricte mais pas recouvrable}
      \end{reponses}
  \end{questionmult}
}


\element{amc}{
  \begin{questionmult}{tpl}\bareme{mz=3}

Laquelle des histoires suivantes correspond à l'exécution effective de cette histoire selon l'algorithme de verrouillage à 2 phases (selon la définition vue en cours : sous-verrous
en lecture partageable et sous-verrous en écriture exclusif, relâchement de tous les verrous lors du \textit{commit}, redémarrage des transactions bloquées dans l'ordre FIFO).
 
      \begin{reponses}
        \mauvaise{\py{i}}
        \bonne{\py{__sol}}
        \mauvaise{\py{j}}
      \end{reponses}
  \end{questionmult}
}


\element{amc}{
\begin{question}{preuve}
  Expliquez pourquoi, si une histoire n'est pas recouvrable, alors elle sera nécessairement modifiée par l'algorithme de verrouillage à deux phases. 
  \AMCOpen{lines=5, dots=false}{\wrongchoice[F]{f}\scoring{0}\wrongchoice[P]{p}\scoring{1}\correctchoice[J]{j}\scoring{4}}
\end{question}
}




\exemplaire{1}{    

%%% debut de l'en-tête des copies :    

%\noindent{\bf Classe d'application d'AMC  \hfill Examen du 01/01/2010}
\begin{center}
{\Large L3 Miage Bases de données}\\
Durée 2h -- documents de CM/TD/TP et dictionnaire bilingue autorisés\\
\end{center}


\vspace{2ex}

\noindent\AMCcode{etu}{8}\hspace*{\fill}
\begin{minipage}{.5\linewidth}
$\longleftarrow{}$ codez votre numéro d'étudiant ci-contre, et écrivez votre nom et prénom ci-dessous.

\vspace{3ex}

\champnom{\fbox{    
    \begin{minipage}{.9\linewidth}
      Nom et prénom :
      
      \vspace*{.5cm}\dotfill
      \vspace*{1mm}
    \end{minipage}
  }}\end{minipage}

\vspace{1ex}



\noindent\hrulefill

\vspace{2ex}

\begin{center}
  Les questions faisant apparaître le symbole \multiSymbole{} peuvent
  présenter zéro, une ou plusieurs bonnes réponses. Les autres ont une
  unique bonne réponse.
\end{center}

\noindent\hrulefill

\vspace{2ex}
%%% fin de l'en-tête

%\melangegroupe{amc}
\restituegroupe{amc}

%\clearpage    

}   


\end{document}
